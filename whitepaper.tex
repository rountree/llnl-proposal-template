% --------------------------------------------------
% LLNL Proposal Whitepaper Template
% --------------------------------------------------
\documentclass[11pt]{article}
\usepackage[letterpaper, margin=1in]{geometry}
\usepackage{times}

% --------------------------------------------------
% Colors
% --------------------------------------------------
\usepackage[table]{xcolor}
\usepackage{tikz}
\usepackage{showframe}

\definecolor{secblue}{HTML}{3B5E7F}
\definecolor{subsecblue}{HTML}{4F81BD}
\definecolor{hdrgray}{HTML}{A7B7C7}

\definecolor{boxblue}{HTML}{587590}
\definecolor{boxgray}{HTML}{E0E6E6}

% --------------------------------------------------
% Draft watermark -- uncomment for drafts
% --------------------------------------------------
%\usepackage[colorspec=.9]{draftwatermark}
%\SetWatermarkText{DRAFT}
%\SetWatermarkScale{1}

% --------------------------------------------------
% Extra packages
% --------------------------------------------------
\usepackage[subfigure]{tocloft}
\usepackage{inconsolata}
\usepackage{abstract}
\usepackage{subfigure}
\usepackage{graphicx}
\usepackage{hyperref}

\hypersetup{
  colorlinks,
  citecolor=violet,
  linkcolor=secblue,
  urlcolor=violet,
  breaklinks
}
\usepackage{wrapfig}
\usepackage{url}
\usepackage{breakurl}
\usepackage{pdfpages}
\usepackage{ctable}
\usepackage{multirow}
\usepackage{appendix}
\usepackage{lipsum}
\usepackage{mdframed}
\usepackage{fancyvrb}
\usepackage{tabu}
\usepackage{tabularx}
\usepackage{multirow}
\usepackage{xspace}

\usepackage{pdfpages}

% Sort multiple citations by citation number
\usepackage[sortcites=true]{biblatex}
\addbibresource{strong-foundations-for-sustainability-fy23.bib}

% Space reduction
\usepackage{caption}
\DeclareCaptionFont{subsecblue}{\color{subsecblue}}
\captionsetup{labelfont={small,subsecblue,bf},font=small}

\usepackage{enumitem}
\setlist{noitemsep,topsep=2pt}

% --------------------------------------------------
% Section heading format
% --------------------------------------------------
\usepackage{titlesec}

%%% \titlespacing{\parameterization}{0em}{.4em}{.5em}
%%% \titlespacing{\section}{0em}{.4em}{.1em}
%%% \titlespacing{\subsection}{0em}{.4em}{.1em}
%%% \titlespacing{\subsubsection}{0em}{.4em}{.1em}
%%% \titlespacing{\paragraph}{0em}{.1em}{.5em}
%%% 
%%% \def\secformat{\sffamily\bfseries}
%%% 
%%% \titleformat{\section}{\large\secformat\color{secblue}}{\thesection}{1em}{}
%%% \titleformat{\subsection}{\large\secformat\color{subsecblue}}{\thesubsection}{1em}{}
%%% \titleformat{\subsubsection}{\normalsize\secformat\color{subsecblue}}{\thesubsubsection}{1em}{}
%%% \titleformat{\paragraph}[runin]{\small\secformat}{\theparagraph}{1em}{}
%%% 
%%% \newcommand{\tablesec}[1]{%
%%%         \vspace{.1em}
%%%         \textbf{\color{subsecblue} #1}\newline}
%%% 
%%% \setlength{\parindent}{0em}
%%% \setlength{\parskip}{.5em}
%%% 
% --------------------------------------------------
% Make TOC fonts match sections
% --------------------------------------------------
\renewcommand\abstractnamefont{\normalsize\secformat\color{secblue}}
\renewcommand\contentsname{Table of Contents}
\renewcommand\cfttoctitlefont{\small\secformat\color{secblue}}
\renewcommand\cftsecfont{\small\secformat\color{secblue}}
\renewcommand\cftsubsecfont{\small\sffamily\color{subsecblue}}
\renewcommand\cftsubsubsecfont{\small\sffamily\color{subsecblue}}

% --------------------------------------------------
% Use \todo and \ncite to add some red text where
% more text or a citation
% --------------------------------------------------
\newcommand{\todo}[1]{{\bf \color{red} #1}}
\newcommand{\ncite}{{\color{red}[cite]}}

% --------------------------------------------------
% ADD YOUR TITLE HERE (it's used in several places)
% --------------------------------------------------
\newcommand{\mytitle}{Performance Portability with Variorum}
\newcommand{\foanumber}{DE-FOA-0002931}
\newcommand{\foaname}{FY23 Funding for Accelerated, Inclusive Research (FAIR)}

% --------------------------------------------------
% Headers & Geometry
% --------------------------------------------------

% Header for other pages
\usepackage{fancyhdr}

% Header for first page
%\fancypagestyle{first}{
%        \lhead{\includegraphics[width=2.7in]{images/llnl-logo.png}}
%        \rhead{LLNL-PROP-XXXXXX\\
%        Program Announcement Number: \foanumber\\
%        \foaname}
%        \cfoot{}
%}


%%% \renewcommand{\headrulewidth}{0pt}
%%% \pagestyle{fancy}
%%% \lhead{
%%%         \begin{tikzpicture}
%%%                 \node[white,draw=white,shape=rectangle,fill=hdrgray] at (-8, 12)
%%%                 {\sf\footnotesize \hbox to \textwidth{
%%%                         \mytitle
%%%                         \hfill
%%%                         FY23 ASCR Proposal}
%%%                 };
%%%         \end{tikzpicture}
%%% }
\chead{}
\rhead{}

% --------------------------------------------------
% Title, Author, Proposal info table
% --------------------------------------------------
\def\lbox{\color{secblue}\small\sffamily\bfseries}
\def\rbox{\small}

%
% Title
\title{\vspace{-.4in}\sf\huge\color{secblue}\mytitle\vspace{-.2in}}

%
% Author Box is actually all the metadata they want
%
\author{
\begin{tabularx}{0.8\textwidth}{r X}
  \lbox Principal Investigator     & \rbox Dr. Suzanne Rivoire, Professor of Computer Science\\
  \lbox Institution                & \rbox Sonoma State University \\
  \lbox Email                      & \rbox \href{mailto:rivoire@sonoma.edu}{rivoire@sonoma.edu} \\
  \lbox Phone                      & \rbox 707-664-3337 \\
  \hline
  \lbox Partner                    & \rbox Dr. Barry Rountree, Computer Scientist \\
  \lbox Institution                & \rbox Lawrence Livermore National Laboratory \\
  \lbox Email                      & \rbox \href{mailto:rountree@llnl.gov}{rountree@llnl.gov} \\
  \lbox Phone                      & \rbox 925-422-3520 \\
  \hline
  \lbox FOA                        & \rbox DE-FOA-0002931 \\
  \lbox Program                    & \rbox ASCR \\
  \lbox Subprogram                 & \rbox Computer Science \\
  \hline
\end{tabularx}
}
\date{}

% --------------------------------------------------
% Main document
% --------------------------------------------------
\begin{document}

\maketitle
%\thispagestyle{first}

\clearpage
\newpage

\section{OBJECTIVES}
This proposal will reinforce research capcity, infrastructure, and expertise
within the Compter Science department at Sonoma State University (SSU).

SSU is one of the 23 campuses of the California State University System
a federally designated Hispanic-serving institution (HSI)
an emerging research institution 

SSU is a MS granting institutions, but CS only grants BS, so exteral collaborations
are critical for engaging undergrad in research 

RESEARCH CAPACITY:

The PI has collaborated in the past with Rountree at LLNL as well as ORNL
LLNL and SSU have had informal collaborations since 2015 including students
being hired as interns (and eventually staff), guest talks, and Rountree teaching
two courses, Dr. Shubbhi Taneja collaborating with Patki.

This grant is necessary for faculty to remain current in the field and act as
co-mentors, rather than simply directing potential student interns to LLNL.

By being able to recruit and fund students in cohorts we can shape teams with
an eye towards diversity and technical growth, as oppposed to one-off approaches
that tends to reward only the most prepared (and usually the most privileged) students.

The structure of having a team work on the technical objectives is important
for promoting a sense of belonging in the research community, which is particularly
important for students selected from under-represented populations.  In addition,
co-mentoring allows students to both imagine a potential future in research as well as
seeing a potential route to get there.  In prior collaborstions with ORNL, 
the PI observed having external mentors increased sense of student ownership.  Having
both the PI and external collaborators present gave them experience and confidence
in sharing their ideas in a professional setting.

(mine the above for a bullet list of objectives)

EXPERTISE:
Students develop facility with Linux development environments, high performance
computing environments, variety of processor architectures, kernel mediation between
hardware and users

INSTITUTIONAL INFRASTRUCTURE:

Structure of the collaboration

\section{TECHNICAL APPROACH}
Evaluate GEOPM and other approaches in the literature (READEX, Adagio, Conductor).

Research and develop the next generation of resource-aware runtime systems that
can handle contemporary scientific workflows (distributed AI/ML, nontraditional
accelerators such as Cerebras, heterogeneous computing, novel storage hierarchies) 
as well as traditional MPI applications.

Primary goal is handling multiple optimization goals (throughput, speed, utilization,
fairness) across multiple constraints (instantaneous power, power variation, total energy, 
temperature) for upcoming workflows and heterogenous architectures.

We see this collaboration as primarily one of building research capacity, infrastructure
and expertise.  While we expect to build upon this work to create an eventual production-
grade runtime system, the expected deliverables will not exceed research prototypes and
proofs of concept.  

Students will:
1. Become familiar with the existng literature
2. Learn how to build and run existing runtime systems in HPC environments
3. Learn how to build and configure benchmarks
4. Learn to think about multidimensional optimization and constrains:
how do design around and leverage tradeoffs depending on the characteristics of the
underlying architecture.
5. Work with diverse architectures (AMD, Intel, nVidia, Arm, IBM Power, Cerebras)
6. Develop competence in low-level system programming (Variorum, msr-safe, NVML, OPAL, ROCM)
7. Work with diverse programming models (MPI, OpenMP, CUDA)
8. Explore multiple scientific workflows that do not necessarily depend on MPI. (Tensorflow, PyTorch)
9. Learn modern linux and HPC development workflows (git, flux, nvml, rocm)
10. Leverage linear programming and other mathematics of optimization

Institutional Capacity and Expertise
\begin{enumerate}
    \item Expect to select six students, 2 sophomore, 2 junior, 2 senior.
    \item New students will be recruited as successful students graduate.
    \item The ability to bring new students up to speed each year will be a demonstration of the new non-physical institutional capacity.
\item Other faculty (inside and outside the computer science department) will be recruited based on their specialties.
\item An enticement to hire additional research-active faculty.  
\item Internships onsite at LLNL give access to career development resources that will benefit wider student population.
\end{enumerate}

% Don't use this.
%For example, the team will be evaluating the suitability of GEOPM to handle a much
%more extensive lists of hardware architectures and use cases.  The result of this
%evaluation will be a report, and should it be deemed feasible, working examples of
%how the underlying software structure could be modified.  We do not expect to be
%submitting polished commits to the GEOPM project; this will be future work on other
%funding streams.

%Put this somewhere
%PI and partner both have an extensive track record teaching, mentoring and hiring
%diverse students.

Runtime able to handle two types of multi-tenancy:  a single user with multiple
jobs running on the same resources as well as multiple users running simultaneous
jobs on a shared resource.  Note the latter will pull in issues of security,
accounting, and fairness.



\begingroup
\section{REFERENCES}
\printbibliography[heading=none]
\endgroup

\end{document}
